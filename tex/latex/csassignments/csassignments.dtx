% \iffalse meta-comment
%
% Copyright (C) 2022 by Alexander Bartolomey
% -------------------------------------------
% MIT License
%
% Copyright (c) 2022
%
% Permission is hereby granted, free of charge, to any person obtaining a copy
% of this software and associated documentation files (the "Software"), to deal
% in the Software without restriction, including without limitation the rights
% to use, copy, modify, merge, publish, distribute, sublicense, and/or sell
% copies of the Software, and to permit persons to whom the Software is
% furnished to do so, subject to the following conditions:
% 
% The above copyright notice and this permission notice shall be included in all
% copies or substantial portions of the Software.
%
% THE SOFTWARE IS PROVIDED "AS IS", WITHOUT WARRANTY OF ANY KIND, EXPRESS OR
% IMPLIED, INCLUDING BUT NOT LIMITED TO THE WARRANTIES OF MERCHANTABILITY,
% FITNESS FOR A PARTICULAR PURPOSE AND NONINFRINGEMENT. IN NO EVENT SHALL THE
% AUTHORS OR COPYRIGHT HOLDERS BE LIABLE FOR ANY CLAIM, DAMAGES OR OTHER
% LIABILITY, WHETHER IN AN ACTION OF CONTRACT, TORT OR OTHERWISE, ARISING FROM,
% OUT OF OR IN CONNECTION WITH THE SOFTWARE OR THE USE OR OTHER DEALINGS IN THE
% SOFTWARE.
% \fi
% \iffalse
%<*driver>
\ProvidesFile{csassignments.dtx}
%</driver>
%    \begin{macrocode}
\NeedsTeXFormat{LaTeX2e}
\ProvidesClass{csassignments}[2022/06/01 v1.0.0 Assignments Class for CS]
%    \end{macrocode}
%
%<*driver>
\documentclass[a4paper]{ltxdoc}
\EnableCrossrefs
\CodelineIndex
\RecordChanges
\usepackage[utf8]{inputenc}
\usepackage[T1]{fontenc}
\usepackage{mathtools}
\usepackage{amsmath}
\usepackage{amssymb}
\usepackage{url}
\usepackage{helvet}
\renewcommand{\familydefault}{\sfdefault}
\begin{document}
  \DocInput{csassignments.dtx}
\end{document}
%</driver>
%    \end{macrocode}
% \fi
%
% \CheckSum{574}
%
% \CharacterTable
%  {Upper-case    \A\B\C\D\E\F\G\H\I\J\K\L\M\N\O\P\Q\R\S\T\U\V\W\X\Y\Z
%   Lower-case    \a\b\c\d\e\f\g\h\i\j\k\l\m\n\o\p\q\r\s\t\u\v\w\x\y\z
%   Digits        \0\1\2\3\4\5\6\7\8\9
%   Exclamation   \!     Double quote  \"     Hash (number) \#
%   Dollar        \$     Percent       \%     Ampersand     \&
%   Acute accent  \'     Left paren    \(     Right paren   \)
%   Asterisk      \*     Plus          \+     Comma         \,
%   Minus         \-     Point         \.     Solidus       \/
%   Colon         \:     Semicolon     \;     Less than     \<
%   Equals        \=     Greater than  \>     Question mark \?
%   Commercial at \@     Left bracket  \[     Backslash     \\
%   Right bracket \]     Circumflex    \^     Underscore    \_
%   Grave accent  \`     Left brace    \{     Vertical bar  \|
%   Right brace   \}     Tilde         \~}
%
% \changes{v1.0.0}{2022/06/01}{Publish class with documentation}
% \changes{v1.0.1}{2022/06/07}{Rename class to not collide with existing assignments class}
%
% \GetFileInfo{csassignments.cls}
%
% \title{The \texttt{csassignments} class}
% \date{}
% \author{Alexander Bartolomey}
% \maketitle
% \tableofcontents
%
% \StopEventually{}    ^^A
%
% \section{Introduction}
% 
% This class file is designed for assignments of university courses, specifically
% in the field of computer science, but can easily be adapted to other fields. It
% provides macros for counting exercise points, adding a grading table, a helpful
% document title block including several non-standard information, and page
% headers with compressed author details. 
% 
% It supports German and English language by default, by requirement of the
% author's university.
% 
% Additional to the layout utilities, the class provides several commonly used
% macros for computer science topics, namely several mathematical operators,
% mathmode utilities and special environments for proofs and theorem. Those are a
% condensed version of macros defined in
% \url{https://github.com/zoomoid/AlphabetClasses}.
% 
% \section{Getting Started}
% 
% Use the class like any default document class 
%
% \begin{verbatim}
%   \documentclass[<option>]{csassignments}
% 
%   \course{}
%   \group{}
%   \due{}
% 
%   \member{}{}{}
%   % ...
% 
%   \begin{document}
%   \maketitle
%   \gradingtable
% 
%   \exercise[<Exercise Points>]{<Exercise Title>}
%   \subexercise{}
%   \end{document}
% \end{verbatim}
%
% Because it inherits from \texttt{article}, you can pass down any options that
% \texttt{article} understands.
%
% \section{Initial Code}
% First, we define the document class and set some initial lengths.
%    \begin{macrocode}
\DeclareOption*{\PassOptionsToClass{\CurrentOption}{article}}
\ProcessOptions\relax
\LoadClass[a4paper,11pt]{article}
\makeatletter
\setlength{\parindent}{0pt}
\setlength{\parskip}{0.6em}
%    \end{macrocode}
% 
% \section{Required Packages}
%
% Notably, the class introduces several dependencies. Here's a list of the
% packages required by \texttt{csassignments}:
%
% \begin{description}
%   \item[geometry] \texttt{CTAN:macros/latex/contrib/geometry}
%   \item[inputenc] \texttt{CTAN:macros/latex/base}
%   \item[babel] \texttt{CTAN:macros/latex/required/babel/base}
%   \item[titlesec] \texttt{CTAN:macros/latex/contrib/titlesec}
%   \item[enumitem] \texttt{CTAN:macros/latex/contrib/enumitem}
%   \item[tocloft] \texttt{CTAN:macros/latex/contrib/tocloft}
%   \item[float] \texttt{CTAN:macros/latex/contrib/float}
%   \item[ifthen] \texttt{CTAN:macros/latex/base}
%   \item[translations] \texttt{CTAN:macros/latex/contrib/translations}
%   \item[csquotes] \texttt{CTAN:macros/latex/contrib/csquotes}
%   \item[fancyhdr] \texttt{CTAN:macros/latex/contrib/fancyhdr}
%   \item[microtype] \texttt{CTAN:macros/latex/contrib/microtype}
%   \item[stmaryrd] \texttt{CTAN:fonts/stmaryrd}
%   \item[graphicx] \texttt{CTAN:macros/latex/required/graphics}
%   \item[pdfpages] \texttt{CTAN:macros/latex/contrib/pdfpages}
%   \item[hyperref] \texttt{CTAN:macros/latex/contrib/hyperref}
%   \item[tikz] \texttt{CTAN:graphics/pgf/base}
%   \item[microtype] \texttt{CTAN:macros/latex/contrib/microtype}
%   \item[amsmath] \texttt{CTAN:macros/latex/required/amsmath}
%   \item[amsthm] \texttt{CTAN:macros/latex/required/amscls}
%   \item[amssymb] \texttt{CTAN:macros/latex/required/amscls}
%   \item[mathtools] \texttt{CTAN:macros/latex/contrib/mathtools}
%   \item[totcount] \texttt{CTAN:macros/latex/contrib/totcount}
%   \item[changepage] \texttt{CTAN:macros/latex/contrib/changepage}
%   \item[etoolbox] \texttt{CTAN:macros/latex/contrib/etoolbox}
%   \item[environ] \texttt{CTAN:macros/latex/contrib/environ}
%   \item[multicol] \texttt{CTAN:macros/latex/required/tools}
%   \item[tgpagella] \texttt{CTAN:fonts/tex-gyre}
%   \item[suffix] \texttt{CTAN:macros/latex/contrib/bigfoot}
% \end{description}
%
%    \begin{macrocode}
\RequirePackage[left=2.5cm,right=2.5cm,top=2cm,bottom=2cm,
  includeheadfoot]{geometry}
\RequirePackage[utf8]{inputenc}
\RequirePackage[english]{babel}
\RequirePackage{titlesec}
\RequirePackage{enumitem}
\RequirePackage{graphicx}
\RequirePackage{tocloft}
\RequirePackage{float}
\RequirePackage{ifthen}
\RequirePackage{translations}
\RequirePackage{csquotes}
\RequirePackage{fancyhdr}
\RequirePackage{microtype}
\RequirePackage{stmaryrd}
\RequirePackage{pdfpages}
\RequirePackage[hidelinks, unicode]{hyperref}
\RequirePackage{tikz}
\RequirePackage{microtype}
\RequirePackage[fleqn]{amsmath}
\RequirePackage{amsthm}
\RequirePackage{amssymb}
\RequirePackage{mathtools}
\RequirePackage{totcount}
\RequirePackage{changepage}
\RequirePackage{etoolbox}
\RequirePackage{environ}
\RequirePackage{multicol}
\RequirePackage{tgpagella}
\RequirePackage{suffix}
\usetikzlibrary{arrows,calc}
%    \end{macrocode}
%
% \section{Translations}
% \texttt{csassignments} comes with several predefined prefixes that are localized in both 
% English and German:
%    \begin{macrocode}
\DeclareLanguage{english}
\DeclareLanguage{german}
\select@language{english}
\ProvideTranslation{german}{prefixAuthor}{Eingereicht von}
\ProvideTranslation{english}{prefixAuthor}{Handed in by}
\ProvideTranslationFallback{prefixAuthor}{Handed in by}
\ProvideTranslation{german}{prefixDate}{am}
\ProvideTranslation{english}{prefixDate}{on}
\ProvideTranslationFallback{prefixDate}{on}
\ProvideTranslation{ngerman}{exercise}{Aufgabe}
\ProvideTranslation{english}{exercise}{Exercise}
\ProvideTranslationFallback{exercise}{Exercise}
\ProvideTranslation{ngerman}{exerciseAbbrev}{Aufg.}
\ProvideTranslation{english}{exerciseAbbrev}{Ex.}
\ProvideTranslationFallback{exerciseAbbrev}{Ex.}
\ProvideTranslation{ngerman}{subexercise}{Teilaufgabe}
\ProvideTranslation{english}{subexercise}{Subexercise}
\ProvideTranslationFallback{subexercise}{Subexercise}
\ProvideTranslation{ngerman}{points}{Punkte}
\ProvideTranslation{english}{points}{Points}
\ProvideTranslationFallback{points}{Points}
\ProvideTranslation{ngerman}{sheet}{Abgabe}
\ProvideTranslation{english}{sheet}{Exercise Sheet}
\ProvideTranslationFallback{sheet}{Exercise Sheet}
\ProvideTranslation{ngerman}{studentId}{Matr.Nr.}
\ProvideTranslation{english}{studentId}{Student ID}
\ProvideTranslationFallback{studentId}{Student ID}
%    \end{macrocode}
%
% \section{Grouping Exercises}
% To support the grading table, we need to initialize several counters for both counting up
% the awarded points of exercises but also for labels of exercises and subexercises.
%    \begin{macrocode}
\newtotcounter{exercisenumber}
\newtotcounter{subexercisenumber}
\newtotcounter{totalpoints}
\edef\exerciseLabelGradingTable{}
\edef\gradingTableExercisePoints{}
\edef\PTableA{}
\edef\PTableB{}
\edef\PTHead{}
\newcommand{\@addPT}[2]{
  \edef\gradingTableExerciseLabel{\GetTranslation{exerciseAbbrev} #1 &}
  \edef\gradingTableExercisePoints{\qquad / $#2$ &}
  \xappto\PTableA{\gradingTableExerciseLabel}
  \xappto\PTableB{\gradingTableExercisePoints}
  \xappto\PTHead{c | }
}
%    \end{macrocode}
% \begin{macro}{\gradingtable}
% You can toggle the displaying of the grading table by adding \cmd{\gradingtable} at some point
% in your document before the first exercise. It requires the compiler to run twice, in order
% expand the control sequence created on the first run, where exercise labels and points were
% appended in order to be combined in a subsequent run.
%    \begin{macrocode}
\newcommand{\gradingtable}{%
  \ifdef{\PTHeadSaved}{%
    {\Large\bfseries Grading Table}\\%
    \begin{table}[h!]
      \renewcommand\arraystretch{2}%
      \begin{tabular}{\expandonce{\PTHeadSaved} c}
        \PTableASaved {\Large$\Sigma$}\\ \hline
        \PTableBSaved \qquad / $\totalpointsSaved$
      \end{tabular}
      \renewcommand{\arraystretch}{1}%
    \end{table}
  }{\texttt{Compile again}\ClassWarning{Compile again to properly 
  display grading table}}
}
%    \end{macrocode}
% \end{macro}
%
% \begin{macro}{\exercise}
% \begin{macro}{\exercise*}
% On top-level, assignments consist of one or more exercise. The
% \cmd{\exercise}\oarg{exercise points}\marg{exercise title} macro provides a
% numbered section title for your assignment. The mandatory argument should
% contain the exercise's title which is then included in the section title, or
% be left blank, in which case nothing is appended to the exercise title. The
% optional argument contains the points rewarded for the exercise in total.
%
% Exercise points are added up in the first run of pdflatex (or your TeX
% backend of choice) and written to the auxilliary file. On the second run,
% the grading table is added to the top of the document, by inclusion of the
% points stored in the first run. \textbf{This is the reason csassignments
% requires two or more runs when used with the grading table}.
%
% If you do not specify points, that is, leave the optional argument field
% empty, the awarded points for the exercise will be set to zero, but the
% exercise will not be included in the grading table.
%
% If you do not specify a title, no empty parentheses are printed.
%
% Using the starred version of the exercise command will allow you to
% explicitly exclude exercise from the grading table (similar to
% \cmd{section*}\marg{section title} not being included in the table of
% contents). Note that you can set points on explicitly excluded exercises.
% The exercise will not count towards the total number of points, but the
% points for the exercise will still be included in the right column of the
% exercise's title line. This might be particularly helpful for modelling
% bonus exercises.
%    \begin{macrocode}
\newcommand{\@pointsPerExercise}[1]{
  \ifstrempty{#1}{}{{\quad\large\bfseries/#1 \GetTranslation{points}}}
}
\newcommand{\@exerciseLabel}[1]{
  \quad\ifstrempty{#1}{}{\bfseries(#1)}
}
\newcommand\exercise[2][]{
  \penalty150
  \ifblank{#1}{\def\points{0}}{\def\points{#1}}
  \stepcounter{exercisenumber}
  \setcounter{subexercisenumber}{0}
  \par
  \vspace{1.5\baselineskip}
  {
    {\Large\bfseries\GetTranslation{exercise}\ \arabic{exercisenumber}}
    \@exerciseLabel{#2}
    \@exerciseRules
    \notblank{#1}{\@pointsPerExercise{#1}}{}
  }
  \par
  \notblank{#1}{
    \addtocounter{totalpoints}{\points}
    \@addPT{\arabic{exercisenumber}}{\points}
  }{}
}
\WithSuffix\newcommand\exercise*[2][]{
  \penalty150
  \ifblank{#1}{\def\points{0}}{\def\points{#1}}
  \stepcounter{exercisenumber}
  \setcounter{subexercisenumber}{0}
  \par
  \vspace{1.5\baselineskip}
  {
    {\Large\bfseries\GetTranslation{exercise}\ \arabic{exercisenumber}}
    \@exerciseLabel{#2}
    \@exerciseRules
    \notblank{#1}{\@pointsPerExercise{#1}}{}
  }
  \par
  \addtocounter{totalpoints}{\points}
}
\AtEndDocument{
  \makeatletter
  \immediate\write\@mainaux{\string\gdef\string\PTableASaved{\PTableA}}
  \immediate\write\@mainaux{\string\gdef\string\PTableBSaved{\PTableB}}
  \immediate\write\@mainaux{\string\gdef\string\PTHeadSaved{\PTHead}}
  \immediate\write\@mainaux{\string\gdef\string\totalpointsSaved%
    {\arabic{totalpoints}}}
  \makeatother
}
%    \end{macrocode}
% \end{macro}
% \end{macro}
% \begin{macro}{\subexercise}
% Subexercises are the subsection to section equivalent for exercises. In fact,
% their counters behave the same way as subsection counters do.
% \cmd{\subexercise}\oarg{Subexercise title} will make a new subexercise under the
% current exercise. Note that point counting for subexercises is not supported.
% Rather, points are only awarded on the exercise level. Therefore, subexercises
% are rather semantical for the exercise's reading flow.
%    \begin{macrocode}
\newcommand{\subexercise}[1][]{
  \penalty150
  \stepcounter{subexercisenumber}
  \par
  \vspace{0.5\baselineskip}
  {
    {\bfseries\GetTranslation{subexercise}\ (\alph{subexercisenumber})}
    \quad\ifstrempty{#1}{}{(#1)}
    \@subexerciseRules
  }
  \par
}
%    \end{macrocode}
% \end{macro}
%    \begin{macrocode}
\def\Vhrulefill{\leavevmode\leaders\hrule height 0.7ex depth
\dimexpr0.4pt-0.7ex\hfill\kern0pt}
%    \end{macrocode}
% \begin{macro}{\exerciseRules}
% \begin{macro}{\subexerciseRules}
% To add line-spanning horizontal rules between exercises and subexercises,
% respectively, for better visual separation, add \cmd{\exerciseRules} add some
% point in the document before the first exercise.
%    \begin{macrocode}
\newcommand{\exerciseRules}{
  \renewcommand{\@exerciseRules}{\hspace{1em}\Vhrulefill} }
  \newcommand{\subexerciseRules}{
  \renewcommand{\@subexerciseRules}{\hspace{0em}\Vhrulefill} }
\newcommand{\@exerciseRules}{}
\newcommand{\@subexerciseRules}{}
%    \end{macrocode}
% \end{macro}
% \end{macro}
%
% \begin{macro}{\noPoints}
% To remove points and grading entirely from the
% document, add \cmd{\noPoints} at some point in the document before the first
% exercise. This overwrites the grading table macro and defines the exercise
% command to not count points.
%    \begin{macrocode}
\newcommand{\nopoints}{
  \renewcommand{\@pointsPerExercise}{}
  \renewcommand{\gradingtable}{}
  \renewcommand{\exercise}{\@exerciseNoPoints}
}
\newcommand{\@exerciseNoPoints}[1]{ 
  \def\points{0}
  \stepcounter{exercisenumber}
  \setcounter{subexercisenumber}{0}
  \par
  \vspace{1.5\baselineskip}
  { {\Large\bfseries\GetTranslation{exercise}\ \arabic{exercisenumber}}
    \@exerciseLabel{#1} \@exerciseRules }
  \par
}
%    \end{macrocode}
% \end{macro}
% \section{Document Title}
% The \texttt{csassignments} class offers several customizations to the default title of article.
% Namely, we can omit the default fields \cmd{\title}, \cmd{\author}, and \cmd{\date}, and 
% replace them with something more fitting to an assignment:
%    \begin{macrocode}
\newcommand{\@course}{Assignment}
\newcommand{\@sheet}{}
\newcommand{\@group}{}
\newcommand{\@prefixId}{\GetTranslation{studentId} }
\newcommand{\@members}{}
\newcommand{\@member}[2][]%
  {\notblank{#1}{#2, \notblank{#1}{\@prefixId{#1}}{} \\}{#2}}
\newcommand{\@due}{}
\newcommand{\@prefixAuthor}{\GetTranslation{prefixAuthor}}
\newcommand{\@prefixDate}{\GetTranslation{prefixDate}}
%    \end{macrocode}
% \begin{macro}{\course}
% \begin{macro}{\sheet}
% \begin{macro}{\group}
% \begin{macro}{\due}
% \begin{macro}{\member}
% Course name with \cmd{\course}\marg{Course}, sheet number with \cmd{\sheet}\marg{sheet number},
% group name with \cmd{\group}\marg{group name}, due date with \cmd{\due}\marg{due date},
% and all members of the group with repeated \cmd{\member}\oarg{student id}\marg{name}.
%    \begin{macrocode}
\newcommand{\course}[1]{\expandafter\notblank\expandafter{#1}%
  {\def\@course{#1}}{}}
\newcommand{\sheet}[1]{\def\@sheet{\GetTranslation{sheet} #1}}
\newcommand{\group}[1]{\renewcommand{\@group}{#1}}
\newcommand{\due}[1]{\renewcommand{\@due}{#1}}
\newcommand{\member}[2][]{
  \expandafter\renewcommand\expandafter\@members\expandafter
    {\@members\@member[#1]{#2}}
  \expandafter\renewcommand\expandafter\pdfmembers\expandafter
    {\pdfmembers\pdfmember[#1]{#2}}
}
%    \end{macrocode}
% \end{macro}
% \end{macro}
% \end{macro}
% \end{macro}
% \end{macro}
%
% \begin{macro}{\PrefixId}
% \begin{macro}{\PrefixAuthor}
% \begin{macro}{\PrefixDate}
% Additionally, you can redefine the prefixes prepended to the individual parts of the title, for example,
% the default (english) prefix for the members line in the title block is ``Handed in by'', which you can
% replace by using \cmd{\PrefixDate}\marg{custom due date prefix}.
%    \begin{macrocode}
\newcommand{\PrefixId}[1]{\renewcommand{\@prefixId}{#1}}
\newcommand{\PrefixAuthor}[1]{\renewcommand{\@prefixAuthor}{#1}}
\newcommand{\PrefixDate}[1]{\renewcommand{\@prefixDate}{#1}}
%    \end{macrocode}
% \end{macro}
% \end{macro}
% \end{macro}
%
%    \begin{macrocode}
\renewcommand{\and}{\\}
\renewcommand{\maketitle}{
  \thispagestyle{empty}
  {\bfseries\Huge\@course}
  \par
  \expandafter\notblank\expandafter\@sheet{{\LARGE\@sheet\\}}{}
  \par
  \expandafter\notblank\expandafter\@members{
    {\footnotesize\bfseries\@prefixAuthor}\\
    \hspace*{4pt}{\large{\vspace*{4pt}\begin{tabular}[t]{@{}l}
    \@members
    \end{tabular}}}\\
  }{}
  \expandafter\notblank\expandafter\@due{
    {\footnotesize\bfseries\@prefixDate}\\
    \vspace*{4pt}{\hspace*{4pt}{\large\@due}}\\
  }{}
  \vskip \baselineskip
  \pagestyle{fancy}
}
\renewcommand{\headrulewidth}{1pt}
\setlength\headheight{4\baselineskip}
\rhead{
  \begin{tabular}{r}
    \textbf{\@course} \\
    \@sheet \\
    {\small\@due}
  \end{tabular}
}
\lhead{
  \small{%
    \begin{tabular}{l}
      \@members
    \end{tabular}%
  }
  }
%    \end{macrocode}
% 
% Lastly, the class handles PDF metadata creation using \texttt{hyperref} from the provided
% macros, i.e., \cmd{\course}, \cmd{\group} etc.
%    \begin{macrocode}
\newcommand{\@pdfauthorprefix}{\expandafter\notblank\expandafter%
  {\@group}{Group \@group: }{}\relax}
\newcommand{\pdfmembers}{}
\newcommand{\pdfmember}[2][]{\expandafter\notblank\expandafter%
  {#1}{#2, {#1};}{#2}}
\AtBeginDocument{\hypersetup{
  pdftitle = {\texorpdfstring{Assignment \@sheet}{}},
  pdfauthor = {\texorpdfstring{\@pdfauthorprefix\pdfmembers}{}},
  pdfsubject = {\texorpdfstring{\@course}{}}
}}
%    \end{macrocode}
%
% \section{Math Helpers}
%
% \texttt{csassignments} includes several frequently used macros for mathmode, that were,
% and still are, helpful with assignment work during several computer science courses.
% Most of them were originally created in one of the \textsc{AlphabetClasses} packages,
% which featured a lot more definitions.
%
% This class only contains a condensed version of them:
%
% \begin{macro}{\N}
% \begin{macro}{\Z}
% \begin{macro}{\R}
% \begin{macro}{\Q}
% \begin{macro}{\C}
% \begin{macro}{\F}
% \begin{macro}{\primefield}
% These macros define common number spaces and fields as double font in mathmode. Notably,
% \cmd{\primefield} makes a double-printed P, as notation for an arbitrary prime field.
%    \begin{macrocode}
\newcommand{\N}{\ensuremath{\mathbf{N}}}
\newcommand{\Z}{\ensuremath{\mathbf{Z}}}
\newcommand{\R}{\ensuremath{\mathbf{R}}}
\newcommand{\Q}{\ensuremath{\mathbf{Q}}}
\newcommand{\C}{\ensuremath{\mathbf{C}}}
\newcommand{\F}{\ensuremath{\mathbb{F}}}
\newcommand{\primefield}{\ensuremath{\mathbb{P}}}
%    \end{macrocode}
% \end{macro}
% \end{macro}
% \end{macro}
% \end{macro}
% \end{macro}
% \end{macro}
% \end{macro}
%
% \begin{macro}{\modring}
% Denoting a ring of integers modulo \(n\) is easily done in short notation with \cmd{\modring},
% see \(\mathbb{Z}/\!n\:\), which even works in regular text due to custom trailing spacing.
%    \begin{macrocode}
\newcommand{\modring}[2]{#1/\!#2\:}
%    \end{macrocode}
% \end{macro}
% \begin{macro}{\derivative}
% The \cmd{\derivative}\marg{variable} prints a typical \(\frac{d}{dx}\) as a shorthand macro.
% Notably it is not even shorter than the explicit fracture, it is just a convenience macro created
% at a time when writing \LaTeX for assignments was supported by semantic macros (a thing \LaTeX
% clearly lacks).
%    \begin{macrocode}
\newcommand{\derivative}[1]{\ensuremath{\frac{d}{d#1}}}
%    \end{macrocode}
% \end{macro}
% 
% \begin{macro}{\matadd}
% \begin{macro}{\matmul}
% \begin{macro}{\matswap}
% These macros are helpful when visualizing matrix operations, namely addition of rows,
% scalar multiplication of a row, and swapping of rows. Arguments of the macros are identical
% to the group operation's parameter.
%    \begin{macrocode}
\newcommand{\matadd}[3]{\ensuremath{\xmapsto{\mathrm{add}_{#1,#2,#3}}}}
\newcommand{\matmul}[2]{\ensuremath{\xmapsto{\mathrm{mul}_{#1,#2}}}}
\newcommand{\matswap}[2]{\ensuremath{\xmapsto{\mathrm{sw}_{#1,#2}}}}
%    \end{macrocode}
% \end{macro}
% \end{macro}
% \end{macro}
% 
% \begin{macro}{\forall}
% \begin{macro}{\exists}
% To make logic quantors look a bit better, we redefine the ``for all'' and ``exists'' quantors with 
% a bit more horizontal space left and right
%    \begin{macrocode}
\renewcommand{\forall}{\ensuremath{\hskip 2pt \oforall \hskip 2pt}}
\renewcommand{\exists}{\ensuremath{\hskip 2pt \oexists \hskip 2pt}}
%    \end{macrocode}
% \end{macro}
% \end{macro}
% \begin{macro}{\floor}
% \begin{macro}{\ceil}
% \begin{macro}{\abs}
% These are some commonly used mathematical functions, that is, rounding up or down, 
% and the absolute function of a value.
%    \begin{macrocode}
\newcommand{\ceil}[1]{\ensuremath{\left\lceil #1 \right \rceil}}
\newcommand{\floor}[1]{\ensuremath{\left\lfloor #1 \right \rfloor}}
\newcommand{\abs}[1]{\ensuremath{\left\vert#1\right\vert}}
%    \end{macrocode}
% \end{macro}
% \end{macro}
% \end{macro}
%
% \begin{macro}{\rfrac}
% For inline math, regular fractions tend to either stretch the line height, 
% resulting in optical heterogeneity of multiple lines, or setting the letters 
% in the fraction so small that they become unreadable. For small, inline fractions,
% \cmd{\rfrac}\marg{numerator}\marg{denominator} offers a less optically noisy option,
% by setting the numerator and denominator in a diagonal arrangement, rahter than purely 
% vertical. Here are some examples: \({}^{1}\!/_{n}\), \({}^{2n}\!/_{k}\), \({}^{100}\!/_{4}\).
%    \begin{macrocode}
\newcommand{\rfrac}[2]{{}^{#1}\!/_{#2}}
%    \end{macrocode}
% \end{macro}
%
% \begin{macro}{\mod}
% The macro defines the usual modulus function with a bit of extra space to the left and right.
%    \begin{macrocode}
\renewcommand{\mod}{\ \mathrm{mod}\ }
%    \end{macrocode}
% \end{macro}
%
% \begin{macro}{\rel}
% To make a new, arbitrary relation, that is, a symbol for with \texttt{amsmath} ensures horizontal
% space left and right using \cmd{\mathrel}
%    \begin{macrocode}
\newcommand{\rel}[1]{\ensuremath{\mathrel{#1}}}
%    \end{macrocode}
% \end{macro}
%
% \begin{macro}{\QED}
% Ending a proof the way tought at RWTH Aachen University, with a square flushed to the right of
% the line, can be done by simply adding \cmd{\QED}.
% Note however that it is semantically more elegant to use \texttt{amsthm} with either predefined
% \texttt{proof} environment, or any custom environment, as these usually also handle adding a label
% to the block, e.g., ``\textbf{Proof}.'' or ``\textbf{Lemma}''.
%    \begin{macrocode}
\newcommand{\QED}{\hfill\(\square\)}
%    \end{macrocode}
% \end{macro}
%    \begin{macrocode}
\newtheoremstyle{it}{3pt}{3pt}{}{}{\itshape}{:}{.5em}{}
\theoremstyle{it}
%    \end{macrocode}
%    \begin{macrocode}
\makeatother
%    \end{macrocode}
%
% \Finale
%
\endinput
